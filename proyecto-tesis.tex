% vim: set fileencoding=utf-8 encoding=utf-8:
\documentclass[11pt,spanish]{article}
\usepackage[utf8]{inputenc}
\usepackage{babel}
\usepackage{fullpage}
\usepackage{url}
\usepackage{times}
%\usepackage{mathrsfs} 
\usepackage{amsmath} 
%\usepackage{amssymb} 
%\usepackage{amsbsy} 
%\usepackage{cancel}
%\usepackage[dvips]{graphicx} 

\newcommand{\reftitle}{\textit}
\newcommand{\vel}{\mathbf{u}}
\newcommand{\vort}{\mathbf{\omega}}
\newcommand{\pos}{\mathbf{x}}

\title{Proyecto de Tesis \\
    Magíster en Ciencias de la Ingeniería Informática \\
    \textsc{Borrador \#1}
}
\author{Roberto Bonvallet}
\date{Primavera de 2008}

\begin{document}
\maketitle
\thispagestyle{empty}
\begin{enumerate}
    \item \textbf{Título del Proyecto de Tesis:}
 
    \item \textbf{Nombre del Alumno:}
        Roberto Javier Bonvallet Carrasco
    \item \textbf{Número de Teléfono:}
        no tiene
         
    \item \textbf{E-mail:}
        \texttt{rbonvall@inf.utfsm.cl}
    \item \textbf{Fecha de ingreso al programa:}
        Primer semestre de 2006
    \item \textbf{Pregrado:}
        Licenciatura en Ingeniería Informática, UTFSM, 2006
    \item \textbf{Profesor guía de tesis:}
        Luis Salinas Carrasco
    \item \textbf{Fecha presentación tema de tesis:}
    \item \textbf{Fecha aprobación tema de tesis:}
    \item \textbf{Fecha tentativa de término:}
    \item \textbf{Comisión interna de graduación:}
\end{enumerate}

\newpage
\section*{Resumen}

\emph{Debe ser suficientemente informativo, y contener una síntesis del proyecto, sus
objetivos,  resultados esperados y palabras claves. }

\section*{Abstract}

\emph{Lo mismo, en inglés}

\newpage
\section{Formulación General de la Problemática y Propuesta de Tesis}

\emph{Debe contener la exposición general del problema, identificando claramente qué
aspectos relacionados con la informática son los más relevantes.  Además,
deberá contener el marco teórico, la discusión bibliográfica con sus
referencias y, finalmente, su propuesta de tesis.}
    
\emph{(La extensión máxima de esta sección es de hasta 5 páginas.  En hojas
adicionales incluya la lista de referencias)}
 

Tesis: implementación de métodos de partículas con componentes
implementados en la GPU, e integrados en un lenguaje de programación
dinámico de alto nivel.

El trabajo de tesis propuesto consiste en la implementación de métodos de
partículas en un lenguaje de programación dinámico de alto nivel, implementando


\subsection{Computación de alto desempeño}

Desde los inicios de la computación, ha existido una demanda creciente de poder
computacional en las áreas de la ciencia y la ingeniería.  A medida que ha
aumentado la capacidad de cómputo disponible, también han crecido el tamaño y la
complejidad de los problemas que se necesita resolver, y en el caso de los
problemas más complejos, no es posible hacerlo en un tiempo razonable.
\cite{parallel-programming}

La computación de alto desempeño (HPC) es la utilización de supercomputadores y
clústers de computadores para resolver problemas computacionales complejos.  Los
conceptos claves en HPC son:

\begin{itemize}
   \item paralelismo:
     la capacidad de ejecutar instrucciones concurrentemente;
   \item escalabilidad:
     la capacidad de acomodarse a problemas de tamaño creciente;
   \item computación commodity:
     sistemas computacionales construídos con hardware
     estándar y de bajo costo, con componentes de software off-the-shelf
     (interoperabilidad).
\end{itemize}

La tendencia actual en sistemas de HPC y en diseño de arquitecturas de hardware
ya no apunta a mejorar el rendimiento de las unidades de cómputo por separado,
sino a explotar el paralelismo para obtener mejoras en el desempeño final de un
programa, a pesar de la carga adicional que significa manejar la concurrencia.
(preferir high throughput sobre high performance)
\cite{hpc-gpu-cuda-slides}

\subsection{Computación en hardware gráfico}
Las unidades de procesamiento gráfico (GPU) son componentes de hardware diseñados
originalmente para acelerar aplicaciones de computación gráfica (CG), que son
intensivas en cómputo y requieren resultados en tiempo real.  Las
características de una aplicación gráfica que pueden ser efectivamente
paralelizadas, y que son explotadas por la GPU, son:

\begin{itemize}
  \item stream processing:
    las operaciones visuales suelen ser independientes entre
    píxeles, y pueden ejecutarse concurrentemente;  destinando procesadores
    separados para cada operación es como las GPUs obtienen la mayor ganancia
    en desempeño
    \cite[\S3.2]{gpupp}

  \item pipelining:
    las diferentes etapas de un algoritmo gráfico (p.ej.
    transformaciones espaciales, aplicación de texturas, rasterización)
    forman una estructura de tubería (pipeline), en que la salida de una etapa
    sirve de entrada a la siguiente;  las GPUs implementan este pipeline en
    hardware, ofreciendo cierto control sobre sus etapas.
    \cite[\S 3.1]{pygpu}

  \item vector processing:
    casi todas las cantidades involucradas en computación
    gráfica son vectores (colores, puntos, segmentos, direccionamiento en
    texturas); por ello, las instrucciones de la GPU operan sobre 4-vectores de
    números de coma flotante, en lugar de sobre cantidades escalares.
    \cite[3.3]{gpupp}
\end{itemize}

En la última década, las GPUs han evolucionado rápidamente en términos de
capacidad computacional y de programabilidad.  La capacidad de cómputo ya supera
por un orden de magnitud la de las CPUs, y 

En la actualidad ya se considera a las GPUs una parte integral del repertorio
de aplicaciones HPC.
\cite{gpu-computing}


\subsection{Programación científica de alto nivel con GPUs}


\subsection{Dinámica de fluídos computacional}
Una de las áreas de investigación que tradicionalmente ha tenido requerimientos
exigentes de poder de cómputo para hacer simulaciones es la dinámica de fluídos.
La dinámica de fluídos computacional tiene importantes aplicaciones industriales
en diversas áreas como la aeronáutica, la meteorología y la geología.

La dinámica de fluídos estudia los fluídos en movimiento.  Los problemas de la
dinámica de fluídos consisten en calcular varias propiedades del fluído (p.ej.
la velocidad $\vel$, la presión $p$, la densidad $\rho$, la temperatura $T$) en
todo el dominio del fluido, durante un intervalo de tiempo.  Estas propiedades
se llaman variables de estado, y están relacionadas por leyes de conservación y
ecuaciones de estado.

En general los problemas de dinámica de fluídos no se pueden resolver
analíticamente excepto en casos triviales.  La dinámica de fluídos computacional
(CFD) usa métodos numéricos para resolver estos problemas, que aun así siguen
siendo complejos:  incluso con ecuaciones simplificadas y usando recursos de
HPC, sólo es posible obtener soluciones aproximadas.

La ecuación más importante de la dinámica de fluídos es la ecuación de
Navier-Stokes, que describe la conservación de momentum en un fluido viscoso,
de manera análoga a la segunda ley de Newton.  
\begin{equation}
    \rho\left(\frac{\partial\vel}{dt} + \vel\cdot\nabla\vel \right) =
    \nu\Delta\vel - \nabla p + \mathbf{f}.
\end{equation}
El término $\rho(\vel\cdot\nabla\vel)$ es denominado aceleración convectiva, y
explica cómo cambia la velocidad con respecto a la posición en un instante de
tiempo fijo.  Este es el único término no lineal de la ecuación, lo que acarrea
problemas numéricos al discretizar el problema, y en la práctica representa el
principal aporte a la turbulencia del fluído.

La turbulencia es el comportamiento caótico observado en los fluídos.  Es un
fenómeno cuya naturaleza no está bien comprendida, y es difícil de cuantificar y
de describir.  Los métodos numéricos para fluídos turbulentos deben incorporar
algún tipo de modelo de turbulencia, generalmente con base experimental, para
dar cuenta de ella.  Además, la turbulencia emerge en varios niveles de escala,
lo que exige alta resolución en la discretización de los problemas, y por lo
tanto más poder computacional.

\subsection{Métodos de partículas}
La mayoría de los métodos de CFD discretizan el dominio del problema utilizando
algún tipo de malla.  Los métodos de partículas (PM), en cambio, son una clase
de métodos que utilizan partículas discretas que arrastran distribuciones
localizadas de variables de estado por el fluído.  Las propiedades físicas del
fluído pueden ser recuperadas como combinaciones lineales de las distribuciones
de todas las partículas.

Los PMs están basados en la forma lagrangiana de las ecuaciones subyacentes.
La forma lagrangiana utiliza como marco de referencia las trayectorias de las
partículas en el tiempo, y en ella no aparece el término de aceleración
convectiva.  Las ecuaciones pasan a ser un sistema de ecuaciones diferenciales
ordinarias:
\begin{align}
    \frac{d\pos_p}{dt} &= \vel_p(\pos_p, t) =
        \sum_{p'} \mathbf{K}(\pos_p, \pos_{p'}; \xi_p, \xi_{p'}) \\
    \frac{d\xi_p}{dt} &=
        \sum_{p'} \mathbf{F}(\pos_p, \pos_{p'}; \xi_p, \xi_{p'})
\end{align}
donde $\xi$ es alguna propiedad del fluído, y $\mathbf{K}$ y $\mathbf{F}$ 
representan la dinámica del sistema físico simulado.

Un tipo importante de métodos de partículas son los métodos de vórtices (VM),
que han sido utilizados desde la década de 1930 para describir la evolución de
estructuras vorticales en fluídos incompresibles~\cite{multiscale}.
En los últimos años se ha demostrado resultados importantes acerca de la
convergencia y 

%Los VMs resuelven la ecuación de Navier-Stokes en su formulación
%velocidad-vorticidad:
%\begin{equation}
%    \frac{\partial\vort}{\partial t} + (\vel\cdot\nabla)\vort =
%    (\vort\cdot\nabla)\vel + \nu\Delta\vort,
%\end{equation}
%donde $\mathbf{u}$ es la velocidad
%y $\vort = \nabla\times\mathbf{u}$ es la vorticidad.  La velocidad se obtiene
%resolviendo la ecuación de Poisson: $\Delta\vel u = -\nabla\times\vort$






\subsection{Propuesta de tesis}



 
 

 

\section{Hipótesis de Trabajo}

\emph{Formule las hipótesis de trabajo señalando claramente su conjetura.}
\emph{(Su extensión no debe exceder el espacio disponible) }
 
 

\section{Objetivos }

\subsection{Objetivos Generales}

\emph{(Su extensión no debe exceder el espacio disponible)}

\begin{itemize}
  \item 
\end{itemize}

\subsection{Objetivos Específicos}

\emph{(Su extensión no debe exceder el espacio disponible) }
       

 

\section{Metodología y Plan de Trabajo}

\emph{(Su extensión no debe exceder el espacio disponible)}

 
\section{Resultados}

\subsection{Aportes y Resultados Esperados}

\emph{(Su extensión no debe exceder el espacio disponible)}
 

\subsection{Formas de Validación}

\emph{(Su extensión no debe exceder el espacio disponible)}

 
 
 

 

\section{Recursos}

\subsection{Recursos disponibles}

\emph{Señale medios y recursos con que cuenta el Departamento de Informática de
la UTFSM, para realizar el proyecto de tesis (libros, software, laboratorios,
etc.).}

\emph{(Su extensión no debe exceder el espacio disponible) }

Los recursos de hardware y software del Laboratorio de Métodos Cuantitativos
serán suficientes para el desarrollo de este trabajo.

\subsection{Recursos solicitados}

\emph{Señale medios y recursos no disponibles en el Departamento de Informática
de la UTFSM, necesarios para realizar el proyecto de tesis (libros, software,
laboratorios, etc. ). Su extensión no debe exceder el espacio disponible}

No hay recursos por solicitar.



\begin{thebibliography}{99}
    \bibitem{gpu-computing}
    John D.~Owens et~al.
    \reftitle{GPU Computing.}
    Proceedings of the IEEE, May 2008.

    \bibitem{parallel-programming}
    Barru Wilkinson, Michael Allen.
    \reftitle{Parallel Programming.}

    \bibitem{hpc-gpu-cuda-slides}
    HPC GPU CUDA slides.

    \bibitem{gpupp}
    Thomas C.~Jansen.
    \reftitle{GPU++: An Embedded GPU Development System for
      General-Purpose Computations.}
    Tesis doctoral, Technischen Universität München, 2007.

    \bibitem{pygpu}
    Calle Lejdfors.
    \reftitle{High-level GPU programming: Domain-specifi optimization and inference.}
    Tesis doctoral, Lund University, 2008.

    \bibitem{multiscale}
    Petros Koumoutsakos.
    \reftitle{Multiscale Flow Simulations Using Particles.}
    Annu.~Rev. Fluid~Mech. 37~(2005) 457--487.

    \bibitem{ppm}
    I.F.~Sbalzarini et~al.
    \reftitle{PPM -~A highly efficient parallel particle-mesh library
      for the simulation of continuum systems.}
    Journal of Computational Physics 215~(2006) 566--588.

    \bibitem{vortex-gpu}
    Diego Rossinelli, Petros Koumoutsakos.
    \reftitle{Vortex methods for incompressible flow simulations on the GPU.}
    Visual Comput~(2008).

    \bibitem{brookgpu}
    Ian Buck et~al.
    \reftitle{Brook for GPUs: Stream Computing on Graphics Hardware.}
    Stanford University.

\end{thebibliography}





%\section*{Abstract}
%Stream computing is a parallel computing paradigm in which large streams of data
%flow through kernel functions, capable of performing non-trivial operations in
%parallel. Streaming architectures achieve high-troughput by exploiting
%algorithms that exhibit high arithmetic intensity.
%
%This paradigm is well-suited for computation on commodity graphic hardware.
%Current graphic processing units are agressively pipelined processors, and are
%capable to perform general-purpose floating-point computation with one order of
%magnitude better performance than CPUs. Vendor-provided APIs are moving from
%supporting only graphic-related operations towards more generic programming
%models.
%
%Still, GPUs still offer a very restricted programming model. Currently there is
%active development in order to integrate streaming low-level APIs into
%higher-level programming languages, including highly dynamic languages. Several
%different approaches have been implemented, allowing to exploit agressive
%parallelism, while keeping the advantages of high-level development.
%
%As opposed to these frameworks, that aim at providing a generic set of operations
%suitable for a wide class of algorithms, this work explores the design of
%high-level stream-computing operations specifically for implementing particle
%methods in fluid dynamics, tailored to tackle the performance bottlenecks of
%this kind of methods.
%
%Particle methods solve transport problems by advecting discrete particles that
%carry the transported quantities, in order to solve the governing equations in
%their Lagrangian form. As all computational fluid dynamics techniques, these
%methods require large computational power, and can enjoy significant speedups by
%exploiting parallelism.
%
%A top-down approach for this specific domain will be adopted, by decomposing
%high-level implementations into parts that are suitable for being parallelized,
%taking into account the perfomance tradeoffs implied with each decision.

\end{document}
% Observaciones y sugerencias
% 1. No esta claro por que hacer esto. Cuales son los beneficios que se esperan.
% 2. ¿Por que las otras propuestas no son adecuadas para el particle methods?
% 3. Me llama la atencion que se plantee una tesis de Doctorado para un prob-
%    lema particular. Que de interesante tiene este problema ?, ¿ es facil ex-
%    tender la solucion a otros problemas en otrosambitos?
% 4. Plantear claramente las hipotesis.
% 5. En la presentacion considerar referencias bibliograficas que avalen la prop-
%    uesta, es decir, que justifiquen la necesiddad (si es que existen).
